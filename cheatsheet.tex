%%%%%%%%%%%%%%%%%%%%%%%%%%%%%%%%%%%%%%%%%
% Ada/SPARK~2014 Mini Cheat Sheet
%
% Original author:
% Martin Becker (https://github.com/mbeckersys)
%
% License:
% The MIT License (see included LICENSE file)
%
%%%%%%%%%%%%%%%%%%%%%%%%%%%%%%%%%%%%%%%%%

\documentclass[10pt]{article}
\usepackage[utf8]{inputenc}
\usepackage[T1]{fontenc}
\usepackage[landscape]{geometry}
\usepackage{xcolor}
\usepackage{listings}
\usepackage{url}
\usepackage{multicol}
\usepackage[
  colorlinks=false, 
  pdftitle={Cheatsheet}, 
  pdfauthor={Martin Becker <martin.becker@tum.de>}, 
  pdfsubject={SPARK~2014 Mini Cheat Sheet}, 
  pdfkeywords={Ada/SPARK, Cheatsheet}
]{hyperref}

%% page margins & headings
\geometry{top=1cm,left=1cm,right=1cm,bottom=1cm}
\pagestyle{empty}

% Don't print section numbers
\setcounter{secnumdepth}{0}
\setlength{\unitlength}{1mm}
\setlength{\parindent}{0pt}

%% Fonts
\renewcommand{\familydefault}{\sfdefault}

%% Color definitions 
\definecolor{Pantone300}{RGB}{  0, 101, 189}
\definecolor{Pantone540}{RGB}{  0,  51,  89}
\definecolor{Pantone301}{RGB}{  0,  82, 147}
\definecolor{Pantone285}{RGB}{  0, 115, 207}
\definecolor{Pantone542}{RGB}{100, 160, 200}
\definecolor{Pantone283}{RGB}{152, 198, 234}
\definecolor{darkgray}{RGB}{ 88, 88, 90}
\definecolor{gray}{RGB}{ 156, 157, 159}
\definecolor{lightgray}{RGB}{ 217, 218, 219}

%% all smaller, please
\makeatletter
\renewcommand{\section}{\@startsection{section}{1}{0mm}%
                                {-1ex plus -.5ex minus -.2ex}%
                                {0.5ex plus .2ex}%x
                                {\normalfont\large\bfseries\color{Pantone300}}}
\renewcommand{\subsection}{\@startsection{subsection}{2}{0mm}%
                                {-1explus -.5ex minus -.2ex}%
                                {0.1ex plus .0ex\nointerlineskip}%
                                {\normalfont\scriptsize\color{Pantone542}}}
\makeatother


%% listing setup
% listings default format
\lstset{captionpos=t, 
  xleftmargin=.2cm,
  basicstyle=\ttfamily\scriptsize,%
  language=Ada,
  commentstyle=\color{gray},
  keywordstyle=\bfseries\color{black},
  keywordstyle=[1]\bfseries\color{black},
  stringstyle=\color{darkgray},
  morekeywords={Dim_Type,Angle_Type,Time_Type,Angular_Velocity_Type,Lat_Type,Val,Pred,Succ,Image,Last,First,Length,Pos,Floor,Ceil,Size},
  identifierstyle=\color{darkgray},
  extendedchars=true,%
 % backgroundcolor=\color{shadecolor},%
  breaklines=true, % Zeilenumbruch
  breakautoindent=true, % Bei Zeilenumbruch einrücken
  tabsize=2, % Breite eines Tabulators
  postbreak=\space,
  showspaces=false, % Keine Leerzeichensymbole
  showtabs=false, % Keine Tabsymbole
  showstringspaces=false,% Leerzeichen in Strings         
  rulecolor=\color{gray},
  frame=l}


\begin{document}
%\footnotesize
\raggedright
{\LARGE\textcolor{Pantone301}{\bf Ada/SPARK~2014 -- Mini Cheat Sheet}}

\begin{multicols*}{3}

\setlength{\premulticols}{1pt}
\setlength{\postmulticols}{1pt}
\setlength{\multicolsep}{1pt}
\setlength{\columnsep}{2pt}

\section{Packages}

\subsection{Specification (*.ads)}
\begin{lstlisting}
package P with SPARK_Mode => On is
   procedure Something (X : Integer);
end P;
\end{lstlisting}
\subsection{Body (*.adb)}
\begin{lstlisting}
package body P with SPARK_Mode => On is
   procedure Something (X : Integer) is
   begin
     -- ..
   end Something;
end P;
\end{lstlisting}

\subsection{Referencing Packages}
\begin{lstlisting}
with P; -- import content of package P
use P;  -- make content of P usable w/o prefix "P."
\end{lstlisting}

\section{Subprograms}

\subsection{With return value}
\begin{lstlisting}
function F1 (X : Integer) return Integer is
   var : constant Integer := X + 1;
begin
   return var;
end f1;
-- same in short ("expression function"):
function F1 (X : Integer) return Integer is (X + 1);
\end{lstlisting}

\subsection{No return value}
\begin{lstlisting}
procedure p1 (Y : in out Natural) is
begin
   Y := F1 (Y);
   Put_Line ("Y is now" & Natural'Image(Y));
end p1;
\end{lstlisting}

\section{Types}
\subsection{Predefined Types}
\begin{lstlisting}
Boolean, Integer, Natural, Positive,
Float, Character, Duration, String
\end{lstlisting}

\subsection{Creating New Types}
\begin{lstlisting}
-- type compatible to predefined Integer:
subtype Months is Integer range 1 .. 12;
-- completely new, Float-incompatible type:
type Bitcoin is new Float;
-- type that wraps around:
type Hours is mod 24; -- ranges from 0 to 23
\end{lstlisting}

\subsection{Array Types}
\begin{lstlisting} 
-- declare type
type Arr_T is array (positive range <>) of Integer;
-- create array variable
A : Arr_T (1 .. 2) := (2, 3);
\end{lstlisting}

\subsection{Composite Types}
\begin{lstlisting}
type My_Vector is record
  x : Float;
  y : Float;
end record;
\end{lstlisting}

\subsection{Enumeration Types}
\begin{lstlisting}
type My_Weekdays is (Monday, Holiday, Friday);
\end{lstlisting}
\columnbreak

\section{Conditional Control Flow}
\begin{lstlisting}
if A then -- ...
elsif B then -- ..
else -- ...
end if;
\end{lstlisting}

\begin{lstlisting}
case weekday is
   when Monday | Friday =>
      Do_Work;
   others =>
      null;
end case;      
\end{lstlisting}

\section{Loops}
\subsection{Counting Loop}
\begin{lstlisting}
for i in Integer range 1 .. 10 loop
  -- ...
end loop;
\end{lstlisting}

\subsection{Iterator Loop}
\begin{lstlisting}
for i in My_Weekdays'Range loop
  -- see first column, also works with other types
end loop;
\end{lstlisting}

\subsection{Head-Controlled}
\begin{lstlisting}
while A > 5 loop
  -- ...
end loop;
\end{lstlisting}

\subsection{Body-Controlled}
\begin{lstlisting}
My_Loop : loop
  A := Calc; -- subprogramm call
  exit My_Loop when A > 5;
end loop My_Loop;
\end{lstlisting}


\section{Attributes}
{\scriptsize\texttt{S} for subtype, \texttt{A} for array. Some of them also work on the instance.}
\begin{lstlisting}[escapechar=\%]
S'First    -- lowest value in range of S
S'Last     -- highest value in range of S
A'First    -- first index of array
A'Last     -- last value of array
A'Length   -- length of array
S'Image(v) -- stringification of value in v
S'Range    -- iterator for loops over type range
A'Range    -- iterator for loops over array indizes
S'Size     -- size in bits of instantiated object
S'Succ(v)  -- value that follows v in type range
S'Pred(v)  -- value that preceded v in type range
S'Val(x)   -- value of type whose position = x
S'Pos(x)   -- position of the value x in the type S
S'Floor(x) -- largest integral value %$\leq$% x in S
S'Ceil(x)  -- smallest integral value %$\geq$% x in S
\end{lstlisting}

\section{Operators}
\begin{lstlisting}
and, or, xor, not -- Logical operators
+, -, *, /, mod, rem, **, abs, =, /=, <, <=, >, >=
\end{lstlisting}

\subsection{Boolean Short-Circuit Operators}
\begin{lstlisting}
if A and then B then ... -- only check B when A true
if A or else B then ... -- only check B when A false
\end{lstlisting}

\columnbreak

\section{Subprogram Contracts}

\subsection{Preconditions}
\begin{lstlisting}
procedure p (X, Y : Integer)
  with Pre => Y /= 0 and then X > 0;
\end{lstlisting}

\subsection{Postconditions}
\begin{lstlisting}
function Increment (X : Integer) return Integer
  with Pre  => X < Integer'Last,
       Post => Increment'Result = X + 1;
\end{lstlisting}
\begin{lstlisting}
procedure Increment (X : in out Integer)
  with Pre  => X < Integer'Last,
       Post => X = X'Old + 1;
\end{lstlisting}

\subsection{Global Variables}
\begin{lstlisting}
procedure P 
  with Global => (Input  => (A, B),
                  Output => (C, D),
                  In_Out => (E));
--  may read A, B and E; and write C, D, E
\end{lstlisting}

\subsection{Information Flow}
\begin{lstlisting}
procedure Sum (A, B : Integer; Result : out Integer)
  with Depends => (Result => (A, B));  
--  Result *must* depend on A and B
\end{lstlisting}

\section{Loop (In)Variants}
\begin{lstlisting}
pragma Loop_Invariant (J in Low .. High));
pragma Loop_Variant (Increases => i, 
                     Decreases => x);
\end{lstlisting}

\section{Testing and Proof}

\subsection{Assertions}
\begin{lstlisting}
pragma Assert (X >= 0); -- abort execution for negative values
pragma Assert (X >= 0); -- can never fail because of previous assert
\end{lstlisting}

\subsection{Assumptions}
\begin{lstlisting}
procedure No_Contract (Y : Integer) is 
begin
  pragma Assume (Y >= 0); 
  -- now analysis only considers positive values
  pragma Assert (Y >= 0); -- never fails in analysis
end No_Contract;
\end{lstlisting}

\subsection{Suppressing False Warnings (Only use with utmost care!)}
\begin{lstlisting}
return A / B;
pragma Annotate (GNATprove, False_Positive,
                 "divide by zero", 
                 "reviewed by John Doe");
\end{lstlisting}
\begin{lstlisting}
X : Integer;
pragma Annotate (GNATprove, Intentional,
                 """X"" is not initialized",
                 "reviewed by John Doe");
\end{lstlisting}




\vspace{\baselineskip}
\linethickness{0.5mm} % Thickness of the footer line
{\color{gray}\line(1,0){30}} % Print the line with a custom color

\footnotesize{
Created by Martin Becker, 2017 \url{<martin.becker@tum.de>}\\				
Released under the MIT license.
}
\end{multicols*}

\end{document}